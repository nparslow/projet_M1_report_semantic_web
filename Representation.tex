\section{La représentation en graphe}

Les fichiers XML sont facilement transférables en représentation en graphe, 
puisqu'ils continennent une hiérarchie d'entrées. En principe chaque niveau de 
l'hiérarchie est représenté par un nœud différent avec des arcs qui lie chaque 
nœud à ses fils dans l'hiérarchie, comme dans 
\hyperref[fig:XMLhierarchy]{Figure~\ref*{fig:XMLhierarchy}}. Cette 
représentation a l'avantage de préserver la structure hiérarchique d'origine.

En plus des arêtes descendantes qui existent dans le schéma 
\hyperref[fig:XMLhierarchy]{Figure~\ref*{fig:XMLhierarchy}}, nous établissons 
des arêtes montantes, afin de retrouver facilement la relation entre un mot qui 
apparaît dans une entrée et le lexème de l'entrée. Les poids sur ces arêtes ne 
sont pas les mêmes que les arêtes descendantes afin de retrouver une différence 
dans ces relations.
