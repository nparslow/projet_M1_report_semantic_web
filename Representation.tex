\section{La représentation en graphe}


\subsection{La première version du graphe}

Les fichiers XML sont facilement transférables en représentation en graphe, 
puisqu'ils continennent une hiérarchie d'entrées. En principe chaque niveau de 
l'hiérarchie est représenté par un nœud différent avec des arcs qui lie chaque 
nœud à ses fils dans l'hiérarchie, comme dans 
\hyperref[fig:XMLhierarchy]{Figure~\ref*{fig:XMLhierarchy}}. Cette 
représentation a l'avantage de préserver la structure hiérarchique d'origine.

En plus des arêtes descendantes qui existent dans le schéma 
\hyperref[fig:XMLhierarchy]{Figure~\ref*{fig:XMLhierarchy}}, nous établissons 
des arêtes montantes, afin de retrouver facilement la relation entre un mot qui 
apparaît dans une entrée et le lexème de l'entrée. Les poids sur ces arêtes ne 
sont pas les mêmes que les arêtes descendantes afin de retrouver une différence 
dans ces relations (Voir [REF] la pondération des relations pour plus de détails).


\subsubsection{Simplification de la structure}
En pratique, il est possible de surpasser d'un grand nombre de nœuds 
intermédiaires dans l'hiérarchie en attribuant une arête directe entre une paire 
de mots dont le poids serait l'addition de toutes les valeurs des liens qui 
constituent le chemin entre les deux mots. Le choix des sommets est un compromis 
entre mettre le plus d'informations possible dans le graphe et veiller à la 
non-explosion de la taille du graphe. Ceci n'est que possible parce que dans les 
tâches effectuées (détaillés dans la partie XXX), nous n'aurons jamais besoin 
d'extraire ces nœuds intermédiaires, même si nous souhaitons tenir compte de 
leur présence.

Ainsi, l'entrée dans \hyperref[fig:banane_full]{Figure~\ref*{fig:banane_full}} 
est simplifiée en l'entrée dans 
\hyperref[fig:banane_simple]{Figure~\ref*{fig:banane_simple}}:

\begin{figure}
\centering
\parbox{5cm}{
\def\svgscale{0.5}
\input{entry_banane_full.pdf_tex}
\caption{}
\label{fig:banane_full}}
\qquad
\begin{minipage}{5cm}
\def\svgscale{0.5}
\input{entry_banane_simple.pdf_tex}
\caption{}
\label{fig:banane_simple}
\end{minipage}
\end{figure}

Notons que nous préservons deux niveaux d'hiérarchie pour le mot d'entrée : un 
contenant le mot seul et l'autre contenant aussi sa catégorie syntaxique. Ceci 
est un moyen d'assurer à ce que le mot soit trouvable dans le réseau sans devoir 
spécifier une catégorie syntaxique particulière, tout en permettant de faire la 
distinction entre plusieurs catégories syntaxiques pour un mot donné. Chaque mot 
taggé à l'intérieur d'une entrée est ainsi lié à une version non-taggée qui 
représente le niveau 'entry' de ce mot, même s'il n'existe pas comme entrée dans 
le dictionnaire de départ.

Le résultat est donc un graphe qui contient deux sortes de nœuds: des mots non-taggés 
(correspondant au niveau entry' et des mots non-taggés (correspondant au niveau 
'pos'), avec des relations pondérées selon le lien établi entre les deux mots dans le dictionnaire.

[IMAGE plein d'arêtes].

\subsection{Graphe minimaliste}


