\section{La structure des dictionnaires utilisés}

Les deux ressources qui seront utilisées sont le Wiktionnaire français en 
format XML du CLLE-ERSS dans le cadre du projet WiktionaryX 
\hyperref[bib:wikixml]{[~\ref*{bib:wikixml}]} et le Littré , qui est aussi 
disponible en format XML \hyperref[bib:littrexml]{[~\ref*{bib:littrexml}]}.

Les informations contenues dans les deux dictionnaires sont similaires. Chaque 
dictionnaire est organisé en entrées, et chaque entrée contient plusieurs 
définitions, des exemples, des synonymes et des informations grammaticales. Le 
wiktionnaire contient en plus d'autres relations sémantiques telles que 
l'antonymie, l'hyperonymie et l’hyponymie.

En termes de style, les deux dictionnaires sont très différents. 

Le Wiktionnaire est une ressource libre qui a comme objectif de décrire tous 
les mots dans toutes les langues. C'est une ressource participative qui peut 
donc être éditée par tout le monde. Les informations sont très structurée et 
destinée à un format informatique, malgré le fait que la ressource soit ouvert 
à l'édition par tout le monde. Il contient un très grand nombre termes, anciens 
comme nouveaux et aussi des formes fléchies.

Le Littré est inspiré du dictionnaire d'Emile Littré du dix-neuvième siècle et 
contient de nombreuses exemples et de citations littéraires. Le dictionnaire, 
n'étant pas destiné au départ à un format en-ligne, est moins structuré en 
termes des informations contenues que le Wiktionnaire et contient aussi moins de 
termes en général. Les informations contenues sont destinées à un 
approfondissent des connaissances, ce qui est reflété parfois dans la manière 
très précise de décrire une entrée, jusqu'à sa catégorie syntaxique. Par 
exemple, l'entrée \lq{F}\rq{} a l'en-tête suivante (contenant sa catégorie 
syntaxique):

\begin{framed}
\xml{entree terme="F"}\newline
\xml{entete}\newline
	\xml{prononciation}
èf, ou, suivant la manière moderne d'épeler, fe 
\xml{/prononciation}
	\xml{nature} s. f. quand on prononce cette lettre èf, une petite f, et 
	s. m. quand on la prononce fe, un f majuscule. \xml{/nature}\newline
\xml{/entete}
\end{framed}

Ceci pose des difficultés pour notre traitement automatique des données, qui 
doivent être prise en compte lors du pré-traitement.

Des extraits des deux dictionnaires se trouvent dans l'annexe. ([REF])

\subsection{Statistiques}

Voici quelques statistiques concernant le Littré et le Wiktionnaire:

\begin{enumerate}
 \item {nombre d'entrées :
	\begin{enumerate}
	 \item Littré :
	 \item Wiktionnaire :
	\end{enumerate}
	}
 \item {nombre de catégories syntaxiques :
	\begin{enumerate}
	 \item Littré :
	 \item Wiktionnaire :
	\end{enumerate}
	}
 \item {nombre de mots-formes :
	\begin{enumerate}
	 \item Littré :
	 \item Wiktionnaire :
	\end{enumerate}
	}
\end{enumerate}
