\section{Pré-traitement}

Dans le but de pouvoir procéder à un traitement homogène des deux dictionnaires, 
une normalisation des deux formats était nécessaire. Ce pré-traitement a permis 
à la fois de supprimer certaines informations qui n'étaient pas utilisée par la 
suite et de convertir les balises et leur contenu en un format plus convenable 
pour notre traitement. Nous réunissons les deux structures dans un seul type de 
fichier XML avec les balises suivantes:

    \begin{itemize}
        \item[$\circ$] entry : un mot du dictionnaire (lexème)
        \item[$\circ$] pos : la catégorie syntaxique (il peut y en avoir 
plusieurs par lexème)
        \item[$\circ$] sense : le niveau hiérarchique représentant plusieurs 
        sens d'un même mot
        \item[$\circ$] subsense : le niveau hiérarchique représentant plusieurs 
        sous-sens d'un même sens global
        \item[$\circ$] def-text : le texte d'une défintion
        \item[$\circ$] semantic : l'emploi sémantique d'un mot (figuratif, 
        absolu etc.)
        \item[$\circ$] register : le registre du mot (familier, vulgaire, 
        soutenu etc.)
        \item[$\circ$] domain : le domaine sémantique du mot (géographie, 
        biologie, culinaire etc.)
        \item[$\circ$] refs : des références vers d'autres entrées lexicales
        \item[$\circ$] examples : des phrases exemples contenant le mot de 
        l'entrée lexicale
        \item[$\circ$] synonyms : synonymes de du mot de l'entrée
        \item[$\circ$] antonyms : antonymes de du mot de l'entrée
        \item[$\circ$] hyperonyms : hyperymes de du mot de l'entrée
        \item[$\circ$] hyponyms : hyponymes de du mot de l'entrée
    \end{itemize}
\bigskip

La DTD des fichiers XML normalisés se trouve dans l'appendice 
\hyperref[App:dtddico]{~\ref*{App:dtddico}}, 
avec les scripts de normalisation 
\hyperref[App:normlittre]{~\ref*{App:normlittre}} et 
\hyperref[App:normwiki]{~\ref*{App:normwiki}}. 
\hyperref[fig:XMLhierarchy]{Figure~\ref*{fig:XMLhierarchy}} illustre 
l'hiérarchie des balises, qui correspond aussi à la structure théorique du 
réseau.

\begin{figure}[!ht]
\centering
\def\svgwidth{\columnwidth}
\input{schema.pdf_tex}
\caption{L'hiérarchie des entrées définie pour les dictionnaires en XML}
\label{fig:XMLhierarchy}
\end{figure}


De légères différences existent entre les structures des deux dictionnaires. 
Tandis que la plupart des informations du Wiktionnaire sont localisées au niveau 
du 'sense', la plupart des informations du Littré sont localisées au niveau du 
'subsense', ce qui ne pose aucun problème pour la construction des deux réseaux.
\newline
\newline
En plus de la restructuration des fichiers XML, les dictionnaires ont été 
nettoyés pour convenir à nos besoins. Certaines balises contiennent trop 
d'informations ou des informations non-pertinentes. Par exemple, les balises des 
commentaires pour le Littré ne contiennent pas de contenu exploitable et les 
citations des textes anciens des mots vieillis qui risquent d'ajouter du bruit 
dans le réseau résultant. Les entrées sont ainsi réduites au schéma ci-dessus 
(\hyperref[fig:XMLhierarchy]{Figure~\ref*{fig:XMLhierarchy}}). Le Littré en 
particulier nécessite une étape importante de nettoyage, puisque des erreurs 
existent dans le balisage des données et un manque d'homogénéité dans la 
structuration des entrées risque de perturber la manière dont laquelle les 
relations sont établies entre les éléments qu'elles contiennent. Les catégories 
syntaxiques utilisées ne sont pas d'une liste énumérable et contiennent des 
descriptions plutôt littéraires. Cette étape de normalisation consistait aussi à 
traduire ces catégories syntaxiques en une liste plus formelle et identifiable.

Les définitions, exemples et autres informations dans les entrées étaient 
ensuite taggées et lemmatisées en utilisant l'outil MElt 
(\hyperref[bib:melt]{[~\ref*{bib:melt}]}). Cette étape est importante pour 
pouvoir relier les mots fléchis des définitions avec leurs lemmes tels qu'ils 
apparaissent dans les entrées et pour savoir de quelle catégorie syntaxique il 
s'agit. Le script XXX qui sert à tagger et lemmatiser les documents se trouvent 
dans l'annexe \hyperref[App:tag]{[~\ref*{App:tag}]}.

\begin{center}
\begin{tabular}{|l|c|c|c|}
\hline
              & totale & bien-taggés & assez-bien-taggés  \\
\hline
tout les mots & 155    & 131 (85\%)  & 139 (89\%)         \\
\hline
mots lexicaux & 87     & 75  (86\%)  & 78  (90\%)         \\
\hline
\end{tabular}
\end{center}

Table (XXXXX) montre que MELT est suffisament accuré d'environ 90\%
du temps. Un tag est `assez-bien-taggé' si notre conversion de tags
élimine la faute. Les mots lexicaux comprennent les verbes, noms,
adjectifs et adverbes, y compris ``être'' et ``avoir''
dans tous leurs formes, mais pas les prepositions.

MELT a des difficultés avec les définitions qui ont un structure
particulier au dictionnaire et donc ne sont pas des phrases complêts
par exemple:

$$Synonyme/ADJ/synonyme \;\; d'/P/de allegro/NC/allegro \;\; ./PONCT/.$$
$$<domain>marine/ADJ/marin$$
où nous voyons
\emph{form\_original / catégorie assigné par MELT / lemma assigné par MELT}
pour
chaque token dans la phrase. Les premiers mots de la phrase sont les
plus souvent affectés. MELT offre la possibilité de s'entrainer sur
un corpus particulier, alors une amélioration possible serait d'annoter
une partie de la dictionnaire manuellement et utiliser cette partie
pour l'entrainement. Il n'est pas clair par contre si le temps pour
faire une quantité suffissament grand vaudrait le gain.
