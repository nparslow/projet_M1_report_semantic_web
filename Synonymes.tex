\section{La Détection de Relations Sémantiques}

Etant donné que le réseau encode de nombreuses relations sémantiques différentes qui sont a priori identifiables par les poids attribués, il devrait être possible dans une certaine mesure d'identifier ces relations ou de mettre certaines relations en avant. Nous avons de nombreux paramètres qui peuvent être modifiés pour changer les valeurs données à certaines relations et l'objectif est de trouver les paramètres qui optimise certaines relations. La relation la plus évident à vouloir chercher est celle de la synonymie, qui est une relation entre deux mots qui ont le même sens (approximativement). Comme suggéré au début, cette relation est en réalité difficilement définissable, surtout par rapport aux nuances de sens que peuvent prendre des mots. La relation entre \lq{content}\rq{} et \lq{joyeux}\rq est certes évidente mais le linguiste motivé peut toujours justifié une différence de sens entre les deux, quoique légère.

Au lieu de rentrer dans les détails du débat de la synonymie, nous jugeons et évaluons nos relations de synonymie trouvées par rapport aux ressources externes, qui encode des relations de synonymie. Il est fort possible que ces ressources ne contiennent pas toutes les relations de synonymie existantes, mais ce sont des avis externes et établis qui fournissent un premier moyen de juger si le réseau sémantique établit bien des relations de synonymie.

\subsection{Détection de synonymes}

L'application est très simple et se base sur le parcours en largeur décrit dans la section [REF]. Elle fait un appel à ce parcours pour chercher les k plus proches voisins d'un mot donné. Ce mot peut être associé à une catégorie syntaxique ou pas. La catégorie syntaxique du mot cible peut également être spécifiée, permettant de chercher éventuellement des relations autre que la synonymie (si les catégories source et cible sont différentes).

La liste des voisins (un maximum de k) est renvoyé, ainsi que le chemin emprunté pour y arriver et la distance entre le mot source et le mot cible.

\subsubsection{L'Interface}

L'interface est réalisée en utilisant les bibliothèques Tkinter [REF] et PMW [REF] sous python.

\begin{center}
\includegraphics[width=13cm]{relationfinderinterface.png}
\end{center}

\begin{enumerate}
    \item{L'utilisateur peut écrire un mot à rechercher ici}
    \item{Le choix de catégorie syntaxique du mot source parmi une liste d'options. L'option ``*" par défaut signifie que le mot recherché peut être de n'importe quelle catégorie}
    \item{Le choix de catégorie des mots cibles. La liste d'options est la même que pour la catégorie source et peut ne pas correspondre à la catégorie source.}
    \item{Le nombre de voisins à renvoyer. Dans le cas où un mot est relié à moins de voisins que demandés, le maximum de voisins sera renvoyé.}
    \item{Pour lancer une recherche du réseau}
    \item{Les voisins sont affichés avec leur distance entre parenthèses. Le chemin est affiché en cliquant sur le signe \lq{+}\rq{}}   

\end{enumerate}

\subsubsection{Optimisation}

L'objectif principal est de maximiser le nombre de mots trouvés qui correspondent à la relation de synonymie. Le fait que les paramètres différents pour la création et le parcours de la matrice (détaillée en section [REF]) soient des valeurs numériques permet d'envisager une optimisation automatique de ces valeurs.

Il s'agit, à partir d'un corpus d'entraînement et un vecteur de paramètres de regénérer la matrice et évaluer les résultats contre les ressources [REF]XXX pour trouver les valeurs qui donnent le meilleur score.

Le vecteur de paramètres comprend les valeurs associées aux types de relations (ex: pos2entry, sense2deftext, ...) et les autres paramètres détaillés dans la section [REF]. La récupération de ce vecteur est incorporé dans le programme pour la création de la matrice, qui contient aussi des méthodes pour changer les valeurs de ces paramètres et pour les réécrire dans les fichiers de configuration.

PARTIE SUR LA FONCTION D'EVALUATION... (je ne veux pas le refaire pour l'instant - Caro - tu peux mettre ta partie ici ?)

L'évaluation selon le corpus se fait en utilisant un métrique défini par nous-mêmes pour juger de la qualité des résultats renvoyés par la recherche. Etant donné que les premiers mots dans la liste sont censés être les plus proches du mot cible, il est important d'incorporer une notion de l'emplacement d'un synonyme trouvé dans la liste de résultats.

DESCRIPTION DU METRIQUE

La bibliothèque Scipy [REF] fournit une méthode \lq{minimize}\rq{} qui sert à minimiser la valeur retourner d'une fonction à partir d'un vecteur de paramètres.

FINIR



\subsection{Evaluation}
