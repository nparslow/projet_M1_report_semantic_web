\section{L'implementation du graphe}

\subsection{Choix de langage et de bibliothèques}
Nous implémentons le graphe en python, pour lequel il existe de nombreuses 
bibliothèques (Scipy, NumPy) efficaces qui permettent de manipuler un grand 
nombre de données numériques. Nous utilisons cElementTree pour traiter les 
fichiers XML et les SparseMatrices de NumPy pour représenter le graphe. Le 
réseau est alors une matrice NxN où N est le nombre de nœuds différents dans le 
graphe. Etant donné le grand nombre d'entrées dans les dictionnaires, il y a un 
avantage clair d'utiliser les matrices sparses, qui permettent de stocker plus 
efficacement un graphe qui contient de nombreux sommets et peu d'arêtes.


\subsection{La création du graphe}
Comme mentionné précedemment, le graphe lui-même est implementé en forme de 
matrice.

\subsection{Wiktionnaire}
