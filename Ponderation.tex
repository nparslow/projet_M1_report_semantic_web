\section{La pondération des relations}

Pour injecter plus de sophistication dans le réseau, la distance entre deux mots 
ne se limite pas au nombre d'arêtes dans le chemin. Les arêtes sont pondérées 
selon le type de nœud sortant et entrant afin de distinguer les relations 
différentes qui peuvent exister à différents endroits de l'entrée. Contrairement 
aux réseaux sémantiques plus sophistiqués tels que WordNet 
\hyperref[bib:wordnet]{[~\ref*{bib:wordnet}]}, qui encodent des relations typées 
spécifiques, notre réseau se base sur un principe plus simple de pondération, 
avec des valeurs réelles assignées par rapport à l'emplacement des mots dans 
l'entrée du dictionnaire.

Les poids des arêtes font partie des paramètres du réseau et sont conservés 
dans un fichier de configuration externe (weight.config), les valeurs étant 
choisies pour optimiser le réseau pour une tâche particulière. L'optimisation 
sera discutée dans la partie [XXX]. Il existe un total de 42 liens différent 
avec un poids différent entre 0 et 1, un lien de 0 étant un 
lien non-existant dans le graphe.

Ci-dessus un exemple d'un extrait de ce fichier, qui se trouve dans l'annexe [REF] :

\begin{framed}
entry2pos = 0.01\newline
pos2entry = 0.01\newline
pos2sense = 0.01\newline
sense2pos = 0.01\newline
pos2syn = 0.01\newline
syn2pos = 0.01\newline
...\newline
sense2subsense = 0.001\newline
subsense2sense = 0.001\newline
subsense2deftext = 0.00001\newline
deftext2subsense = 0.00001\newline
subsense2semantic = 0\newline
\end{framed}

En plus de ces paramètres basés uniquement sur la structure des entrées, 
d'autres paramètres sont utilisés pour faire ressortir des informations 
lexicales en plus de ces relations déjà trouvées. Ces paramètres se trouvent 
dans un deuxième fichier de configuration, qui peut être modifié de la même 
façon que le fichier de poids. La différence est que l'utilisateur peut 
spécifier le type d'opération mathématique (ex: multiplication, addition etc.) à effectuer sur le poids d'origine. Chaque paramètre est écrit sur une seule ligne avec l'étape de la création du graphe à 
laquelle le paramètre sera implémenté, ainsi que d'autres options utilisées pour 
spécifier comment le paramètre peut être utilisé.

Etant donné un mot x et un mot y, où x est le mot d'entrée et y le mot de sa 
définition: le poids original entre x et y serait le poids spécifié entre un mot d'entrée et 
sa définition (c'est-à-dire entry2pos + pos2sense + sense2subsense + 
subsense2deftext). Ensuite ce poids peut être transformé de plusieurs façons:

\subsection{En fonction des propriétés catégorielles}
Pendant l'étape de préparation du graphe, avant la création de la matrice, les 
poids sont modifiés en fonction des propriétés catégorielles pour donner plus 
ou moins de poids à certaines catégories. Les paramètres sont de la forme 
suivante :
    \begin{framed}
        POS\_de\_y, POS\_de\_x, fichiers\_contextes, opération\_math, valeur
    \end{framed}

\subsubsection{Modification uniquement en fonction de la catégorie 
syntaxique du mot y.}
Ces paramètres servent à privilégier certaines catégories de mot par rapport à 
d'autres, en supposant que certaines catégories sont plus utiles pour établir 
des relations sémantiques. Ceci se fait en fonction des relations que nous souhaitons faire ressortir. Par exemple, il peut être plus pertinent de prendre en compte les adjectifs si ce sont des descriptions qui doivent ressortir et de donner plus d'importance aux verbes si ce sont des actions qu'il faut caractériser (une plus grande importance se traduisant par un plus petit poids). Un paramètre est spécifié pour 
chaque catégorie syntaxique afin d'établir des modifications relatives entre 
catégories. Par exemple, pour la catégorie \lq{adjectif}\rq: \lq{prepare: Adj, 
None, None, +, 0.6}\rq{} signifie que 0.6 est ajouté à tout noeud qui va vers un 
mot de catégorie \lq{Adj}\rq{}.

\subsubsection{Modification en fonction de la catégorie syntaxique du mot x, du 
mot y et d'un fichier de contextes spécifiques.} 
Jusqu'à présent le poids des relations a été fondé uniquement sur l'emplacement 
des mots y dans l'entrée et de leurs catégories syntaxiques. Mais il est aussi 
important d'analyser la structure des informations à l'intérieur des 
définitions, puisque les définitions font preuve d'une régularité qui permet de 
donner plus de poids à certains mots. Sans faire une analyse syntaxique complète 
des définitions, nous visons à cibler certains mots qui apparaissent dans un 
certain contexte afin de renforcer la relation entre le mot de l'entrée et ces 
mots cibles. La forme du paramètre dans le fichier de configuration est la même 
que pour les paramètres précédents, sauf que la deuxième option 
lq{POS\_de\_x}\rq{} est spécifiée, ainsi que le chemin vers le fichier de 
contextes légitimisants. Vu que les contextes sont des expressions qui aident
seulement à la définition, les mots lexiques qui font partie d'un contexte
retrouvé sont ignoré dans la construction du graphe. 

Nous jugeons nécessaire de spécifier la catégorie syntaxique du mot x ainsi que 
la catégorie du mot y. Ceci parce que la forme de la définition dépend en grande 
partie de la catégorie syntaxique du mot de l'entrée. La relation entre un 
mot x de catégorie \lq{nom}\rq{} et un mot y de catégorie \lq{nom}\rq{} n'est pas la même 
relation que celle entre un mot x de catégorie \lq{nom}\rq{} et un mot y de 
catégorie \lq{verbe}\rq.

Par exemple, pour un mot x de catégorie \lq{adverbe}\rq, il pourrait être 
pertinent de renforcer le lien entre x et y, si y est un adjectif qui apparaît 
dans le contexte \lq{d'une manière mot\_y}\rq.

\underline{Syntaxe}\newline
Nous définissons une syntaxe particulière pour représenter les contextes 
légitimisants. En général ce sont des séquences de mots précédant et/ou suivant 
le mot y en question, qui signale que ce mot doit être mis en valeur.\newline

\begin{itemize}
    \item{Les contextes précédents et suivants sont des séquences de mots. Ces 
    séquences peuvent contenir soit des mots-formes, soit des lemmes, ce qui 
    permet plus de flexibilité dans la rédaction des règles et de ne pas devoir 
    expliciter toutes les combinaisons de formes fléchies possibles.}

    \item{\# représente la position du mot y, ce qui permet de spécifier le 
    contexte précédent et le contexte suivant du mot.}

    \item{Les symboles \string^ et \$ représentent respectivement le début et 
    la fin de la phrase}

    \item{Un chiffre indique le nombre maximum de mots qui peuvent apparaître à 
    une position donnée, sans devoir spécifier la forme ou le lemme de ces mots. 
    Notez que ces mots ne peuvent pas être un signe de ponctuation ou de la 
    même catégorie syntaxique que le mot y. Ceci fait en sorte que ce soit 
    le premier mot de la catégorie recherchée qui est renforcée par ce 
    contexte.}

\end{itemize}

Par exemple, dans l'entrée du mot \lq{langoureusement}\rq{}, la définition 
\lq{d'une manière langoureuse}\rq{} et le contexte légitimisant \lq{de une manière 
\#}\rq{} permettrait de renforcer la relation entre \lq{} langoureusement\rq{} et \lq 
langoureux\rq{}, ce dernier apparaissant dans la position spécifiée par \#. Dans 
cet exemple, il n'y a pas de contexte suivant spécifié (puisqu'aucun mot ne suit 
le \#) et ceci est le cas dans la plupart des contextes, où le contexte 
précédent est le plus important. Pour simplifier les contextes, nous permettons 
de ne pas exprimer le \# et dans ce cas, la séquence de mots est considérée 
comme la séquence précédente du mot y. Pour reprendre l'exemple précédent, le 
contexte \lq{de une manière \#}\rq{} peut également être exprimé \lq{de une 
manière}\rq. Notons que si un contexte suivant est explicité, le \# est 
nécessaire.

Ci-dessous quelques exemples de contextes pour une paire de catégories pour les 
mots x et y:

\begin{table}[ht]
\centering
\begin{tabular}{|p{1cm}|p{1cm}|p{5.5cm}|p{8cm}|}
\hline
POS de x & POS de y & Contexte & Description\\[0.5ex]
\hline
Adj & Adj & . \# . & Un adjectif entre deux points \\
Nom & Nom & \string^ \# ,  & Un nom au début de la définition, suivi d'une 
virgule \\
Nom & Nom & synonyme de & Un nom qui suit la séquence \lq{synonyme de}\rq \\
Adj & Verbe & en parlant de une personne qui 3 & Un verbe qui suit la séquence 
\lq{en parlant de une personne qui}\rq avec un maximum de 3 mots intervenants 
qui ne sont pas de catégorie PONCT ou V \\
Adj & Nom & qui a des rapports avec 2 & Un nom qui suit la séquence \lq{qui a 
des rapports avec}\rq avec un maximum de 2 mots intervenants qui ne sont pas de 
catégorie PONCT ou N \\ [1ex]
\hline
\end{tabular}
\caption{Quelques exemples de contextes légitimisants}
\label{table:nonlin}
\end{table}


Les contextes ont été choisis en analysant les entrées de dictionnaire pour 
chaque type de catégorie syntaxique et en remarquant les régularités. Quelques 
contextes sont applicables pour un nombre de catégories différentes, par exemple 
\lq{synonyme de}\rq, \lq{hyperonyme de}\rq, et d'autres sont plus spécifiques 
aux contextes.

Quelques abnormalités d'étiquetage syntaxique obligeait aussi de tenir compte de certaines formes qui peuvent paraître bizarre. Par exemple \lq{se dit des}\rq est étiqueté de la façon suivante:

\begin{framed}
se/ClR/clr dit/V/dire des/DET/un
\end{framed}

où on s'attendre à voir le lemme \lq{de}\rq{} pour \lq{des}\rq{}. Par conséquence, cette expression ne correpond pas au contexte \lq{se dit de}\rq{} et nous sommes obligés de spécifier une deuxième règle de la forme \lq{se dit des}\rq{} ou \lq{se dit un}\rq{} pour ces cas anormaux. Une autre possibilité est d'exprimer ces contextes avec le seul contexte \lq{se dit 1}\rq{} où la forme et le lemme du mot qui suit \lq{dit}\rq{} ne sont pas explicités.

\underline{L'implémentation en Python}\newline
La correspondance entre les contextes d'un mot à l'intérieur d'une définition et 
les contextes légitimisant se fait par des expressions régulières, qui sont 
générées à partir des contextes des fichiers et utilisé pour matcher les contextes
précédents et suivant.

Chaque contexte est un tuple contenant une expression régulière pour le contexte 
précédent et une expression régulière pour le contexte suivant.

Les séquences sont transformées de la façon suivante:
\begin{itemize}
    \item{Toute la ligne est divisée en deux sur le symbole \#. S'il n'existe 
    pas de symbole \#, toute la séquence est le contexte précédent est le 
    contexte suivant est nul.}
    \item{Pour chaque contexte (précédent et suivant), l'expression régulière 
    est générée:}
    \begin{itemize}
        \item{tout caractère qui est un caractère spécial pour les expressions 
        régulières doit être échappé (sauf \^ et \$ qui retiennent leur 
	propriété de caractères spéciaux)}
        \item{tout mot est converti en une entité taggée et lemmatisée tel que 
        le mot peut être considéré comme le mot-forme ou le lemme du mot. Par 
	exemple \lq{mot}\rq  est transformé en ((mot/[\string^ /]+?/[\string^ 
	/]+?)|([\string^ /]+?/[\string^ /]+?/mot)) }
    \end{itemize}
\end{itemize}

\begin{figure}
\centering
\parbox{17cm}{
\def\svgscale{0.6}
\input{regexExample.pdf_tex}
\caption{La construction d'une expression régulière pour un contexte précédent d'un mot.}
\label{fig:regex}}
\end{figure}

\subsection{Modification en fonction de l'emplacement du mot dans une définition}
Ce paramètre permet de diminuer le poids des premiers mots d'une définition, en supposant que ce sont les termes les plus importants pour représenter l'entrée. La valeur à rentrer est le nombre de mots au début de la définition qui recevront une valeur diminuée. Le principe est de multiplier chaque arête dans le graphe par une certaine valeur multiplicatrice, qui commence très petite et augmente jusqu'à la valeur maximale de 2. Cette valeur maximale de 2 sera utilisée pour tous les mots qui suivent ces premiers mots mis en avant. L'accroissement de la valeur multiplicatrice suit une pente avant d'arriver à cette valeur maximale.

Par exemple, si le nombre de mots (n) mis en avant et de quatre, les quatre premiers mots seront multipliés par un plus petit nombre que les mots suivants. Le premier mot sera multiplé par la plus petite valeur, le deuxième mot par la deuxième plus petite valeur et ainsi de suite, jusqu'à la position n+1, où la multiplication reste constante à 2.


\begin{figure}
\centering
\parbox{5cm}{
\def\svgscale{1}
\input{multiplyFirstWords.pdf_tex}
\caption{L'accroissement de la valeur multiplicatrice jusqu'à la valeur maximale de 2 pour diminuer le poids des quatre premiers mots de la définition.}
\label{fig:multiply}}
\end{figure}

\subsection{Modification en fonction de la catégorie syntaxique recherchée}
Une autre option est de donner plus d'importance (en forme d'une valeur diminuée) aux mots qui sont de la même catégorie qu'un mot recherché dans le réseau. Ceci peut être important pour la recherche de synonymes, où nous souhaitons privilégier les relations entre mots de la même catégorie. Cette étape n'est pas implémentée dans la création de la matrice, mais plutôt pendant la recherche du graphe (Voir Parcours en largeur [REF]).







