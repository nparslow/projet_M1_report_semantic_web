\section{La Conception du Graphe}

La construction d'un réseau sémantique à partir d'un dictionnaire est possible 
grâceà la structure des dictionnaires qui permet de faire ressortir des liens 
sémantiques.La structure en sens, sous-sens, définitions, et exemples, parmi 
d'autres informations,mais aussi la structure interne des définitions contient 
une régularité qu'il estimportant d'exploiter au maximum. Nous tenons donc à 
conserver le plus possible cettestructure dans la transformation de dictionnaire 
en graphe. Un graphe est définiformellement comme un ensemble de sommets et un 
ensemble d'arcs qui relient une pairede sommets.

\[
G = <S, A>
\]

Pour représenter un dictionnaire par un graphe, nous considérons que les 
sommets peuvent être les mots individuels du dictionnaire ou même les niveaux 
intermédiaires de la structure tels que 'exemple', 'définition', 'synonyme', 
'antonyme' etc. Les arcs sont alors les liens qui lient les différents éléments 
d'une entrée de dictionnaire et permettraient de trouver un lien entre un 
lexème donné et la manière dont il est décrit dans son entrée du dictionnaire.

\subsection{Remarques sur le vocabulaire}
Nous appelons 'mot' toute unité minimale du lexique. Un mot peut être
soitfléchi, soit non-fléchi et par défaut nous faisons référence aux mots 
non-fléchis sous leur forme de dictionnaire. Par principe, nous restreignons le 
réseau aux lemmes, mais il est possible qu'il y apparaît des formes fléchies en 
cas de non-identification du lemme.

Par 'entrée' de dictionnaire nous faisons référence à un groupe d'informations 
(catégories syntaxiques, sens, définitions, exemples etc.) associées à un mot 
donné. Par conséquent, le terme 'entrée' peut aussi être utilisé pour dénoter 
le mot lui-même, et par extension les informations contenues pour ce mot donné.

La relation sémantique de synonymie est définie entre deux termes de la même 
catégorie de discours qui ont le même sens et qui peuvent donc être substitués 
l'un pour l'autre sans modifier le sens de la phrase. Cette définition pose 
évidemment des problèmes, surtout à cause du fait qu'il est toujours possible 
de trouver une différence de sens ou d'usage entre deux mots malgré le fait 
qu'ils soient habituellement classés en synonymes. Il est parfois souhaitable 
de parler de proche-synonymes au lieu de synonymes tout court. Néanmois, 
nous préférons utiliser le terme 'synonyme' pour parler de ces cas, sans 
postuler de théorie sur les frontières de la synonymie. Par la suite, la 
synonymie sera définie en termes de relations attestées dans des ressources 
externes et nous nous reportons à ces références pour établir si deux mots sont 
en relation de synonymie ou pas.

De même pour les relations d'antonymie, d'hyperonymie et d'hyponymie. 
L'antonymie est définie comme la relation entre deux mots à sens opposé. 
L'hyperonymie entre un mot dont l'extension contient l'extension d'un autre mot 
(par exemple, 'véhicule' est l'hypernym de 'voiture'). L'hyponymie est la 
relation inverse d'hyperonymie, entre un mot dont l'extension est incluse dans 
l'extension d'un autre (pour reprendre le même exemple, 'voiture' est un 
hyponyme de 'véhicule').

[AUTRES DEFINITIIONS...]
