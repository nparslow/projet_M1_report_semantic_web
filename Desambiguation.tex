\section{Désambiguation}
\subsection{Programme}

La deuxièmme tâche que nous nous sommes donnée est la désambiguisation des mots
en contexte. Cela consiste à calculer la distance entre, d'une part, une
phrase entrée par l'utilisateur et, d'autre part, chaque définition du mot à
disambiguïser. Le programme va alors calculer un vector pour chaque couple
(phrase / définition) et la définition renvoyée sera celle ayant le vecteur le
plus important.



\subsection{Evaluation}

L'évaluation de ce programme a été faite en deux temps. Tout d'abord 
l'évaluation grâce au corpus RomansEval et ensuite à l'aide d'un corpus plus 
petit que nous avons nous-même réalisé.

\subsubsection{RomansEval}

Le RomansEval est un corpus constitué d'environ 45000 phrases. De plus, 
certains mots présents dans ce corpus sont ambigus. Ces mots sont regroupés dans 
trois fichiers HTML (un fichier par catégorie syntaxique : Adj, N, V) qui 
contiennet en plus leurs différents définitions. Et pour finir, ils sont aussi 
répertoriés dans un fichier de référence avec diverses informations 
supplémentaires telles que :

\begin{enumerate}
 \item la catégorie syntaxique du mot
 \item son lemme
 \item le numero de la phrase dans laquelle il apparait
 \item l'occurrence présente dans la phrase
 \item les différents votes des personnes qui ont fait ce corpus
 \item la définition qui eu le nombre de voix le plus élevé
 \end{enumerate}

Pour réaliser notre évaluation, nous avons mis au point un script qui, pour 
chaque fichier HTML, le lit et récupère les informations pertinentes dans un 
dictionnaires python. Ensuite le fichier de référence est lu ligne par ligne 
ce qui permet de connaître chaque occurrence, son contexte et les définitions 
liées à ce lemme.

Malheureusement pour nous, ce corpus étant beaucoup trop important, nous 
n'avons pas pu nous servir de toutes les données qu'il contient. Pour chaque 
lemme nous avons analysé les 20 premières occurrences pour avoir un score 
d'exactitude le plus précis possible.

Le score obtenu est un peu différent selon la catégorie observée:

\begin{enumerate}
 \item pour les adjectifs, le score d'exactitude est de 20%
 \item pour les noms, le score d'exactitude est de 15%
 \item pour les verbes, le score d'exactitude est de 25%
\end{enumerate}

Nous avons conscience que ces score sont mauvais mais celà peut en partie 
s'expliquer en étudiant de plus près les définitions. En effet, si l'on regarde 
par exemple le lemme FRAIS. Il possède, dans le RomansEval, huit définitions 
différentes. Et parmi ces huit définitions certaines sont très proches et la 
différence entre elles est difficile à percevoir, même pour un être humain. Par 
exemple : `Qui est légèrement froid ou qui procure une sensation de froid 
léger.' et `Qui est empreint de froideur, dépourvu de cordialité.'

\subsubsection{Notre corpus}


