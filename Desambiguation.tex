\section{Désambiguation}
\subsection{Programme}

La deuxièmme tâche que nous nous sommes donnée est la désambiguisation des mots
en contexte. Cela consiste à calculer la distance entre, d'une part, une
phrase entrée par l'utilisateur et, d'autre part, chaque définition du mot à
disambiguïser. Le programme va alors calculer un vector pour chaque couple
(phrase / définition) et la définition renvoyée sera celle ayant le vecteur le
plus important.



\subsection{Evaluation}

L'évaluation de ce programme a été faite en deux temps. Tout d'abord 
l'évaluation grâce au corpus RomansEval et ensuite à l'aide d'un corpus plus 
petit que nous avons nous-même réalisé.

\subsubsection{RomansEval}

Le RomansEval est un corpus constitué d'environ 45000 phrases. De plus certains 
mots présents dans ce corpus sont ambigus. Ces mots sont regroupés dans un 
fichier HTML qui contient en plus leurs différents définitions. Et pour 
finir, ils sont aussi répertoriés dans un fichier de référence avec diverses 
informations supplémentaires telles que:

\begin{enumerate}
 \item la catégorie syntaxique du mot
 \item son lemme
 \item le numero de la phrase dans laquelle il apparait
 \item l'occurrence présente dans la phrase
 \item les différents votes des personnes qui ont fait ce corpus
 \item la définition qui eu le nombre de voix le plus élevé
 \end{enumerate}


 
\subsubsection{Notre corpus}


