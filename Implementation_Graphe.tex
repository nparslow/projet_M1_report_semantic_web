\section{L'implementation du graphe}

\subsection{Choix de langage et de bibliothèques}
Nous implémentons le graphe en python, pour lequel il existe de nombreuses 
bibliothèques (Scipy, NumPy) efficaces qui permettent de manipuler un grand 
nombre de données numériques. Nous utilisons cElementTree pour traiter les 
fichiers XML et les SparseMatrices de NumPy pour représenter le graphe. Le 
réseau est alors une matrice NxN où N est le nombre de nœuds différents dans le 
graphe. Etant donné le grand nombre d'entrées dans les dictionnaires, il y a un 
avantage clair d'utiliser les matrices sparses, qui permettent de stocker plus 
efficacement un graphe qui contient de nombreux sommets et peu d'arêtes.

\subsection{Simplification de la structure}
En pratique, il est possible de surpasser d'un grand nombre de nœuds 
intermédiaires dans l'hiérarchie en attribuant une arête directe entre une paire 
de mots dont le poids serait l'addition de toutes les valeurs des liens qui 
constituent le chemin entre les deux mots. Le choix des sommets est un compromis 
entre mettre le plus d'informations possible dans le graphe et veiller à la 
non-explosion de la taille du graphe. Ceci n'est que possible parce que dans les 
tâches effectuées (détaillés dans la partie XXX), nous n'aurons jamais besoin 
d'extraire ces nœuds intermédiaires, même si nous souhaitons tenir compte de 
leur présence.

Ainsi, l'entrée dans \hyperref[fig:banane_full]{Figure~\ref*{fig:banane_full}} 
est simplifiée en l'entrée dans 
\hyperref[fig:banane_simple]{Figure~\ref*{fig:banane_simple}}:

\begin{figure}
\centering
\parbox{5cm}{
\def\svgscale{0.5}
\input{entry_banane_full.pdf_tex}
\caption{}
\label{fig:banane_full}}
\qquad
\begin{minipage}{5cm}
\def\svgscale{0.5}
\input{entry_banane_simple.pdf_tex}
\caption{}
\label{fig:banane_simple}
\end{minipage}
\end{figure}

Notons que nous préservons deux niveaux d'hiérarchie pour le mot d'entrée : un 
contenant le mot seul et l'autre contenant aussi sa catégorie syntaxique. Ceci 
est un moyen d'assurer à ce que le mot soit trouvable dans le réseau sans devoir 
spécifier une catégorie syntaxique particulière, tout en permettant de faire la 
distinction entre plusieurs catégories syntaxiques pour un mot donné. Chaque mot 
taggé à l'intérieur d'une entrée est ainsi lié à une version non-taggée qui 
représente le niveau 'entry' de ce mot, même s'il n'existe pas comme entrée dans 
le dictionnaire de départ.

Le résultats est un graphe qui contient deux sortes de nœuds des mots non-taggés 
(correspondant au niveau entry' et des mots non-taggés (correspondant au niveau 
'pos').

[IMAGE plein d'arêtes].

\subsection{La création du graphe}
Comme mentionné précedemment, le graphe lui-même est implementé en forme de 
matrice.

